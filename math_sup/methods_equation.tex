\documentclass[a4paper,twoside,openright]{article}
\usepackage{amsmath}
\begin{document}



\section*{Rebound calculation}
Formally, the homoeostatic rebound $H_i$ of an individual $i$ was expressed as:

\begin{align}
H_i &=  R_i - \hat{R_i} \\
\hat{R_i} &= \alpha + \beta{} B_i
\end{align}


Where,
\begin{itemize}
	\item $\hat{R}$ is the \emph{predicted} sleep \emph{after} treatment ($ZT \in [0, 3]$),
	\item $R$ is the \emph{measured} sleep \emph{after} treatment ($ZT \in [0, 3]$),
	\item $B$ is the sleep measured \emph{before} treatment ($ZT \in [0, 3]$), and
	\item $\alpha$ and $\beta$ are the coefficients of the linear regression $R_C = \alpha + \beta{B_C}$ on the control group $C$.
\end{itemize}

\begin{align}
\alpha &=  \bar{R_C} - \beta\bar{B_C} \\
\beta &= \frac{Cov(R_C, B_C)}{Var(B_C)}
\end{align}


\section*{Behavioural state}

\begin{align}
B = 
\begin{cases}
quiescence,            & \text{if } V_{max} < T_V \forall i\\
micro\text{-}movement, & \text{if } \sum^{i}{|X_i - X_{i-1}|} < T_d\\
walking,               & \text{otherwise}
\end{cases}
\end{align}



Where,
\begin{itemize}
	\item $V_{max}$ is the maximal velocity,
	\item $T_V$ the validated threshold under which immobility is scored,
	\item $X$ is the position along the tube,
	\item $T_d$ is a threshold of 15 mm on the distance moved above which $walking$ is scored.
\end{itemize}

The $T_d$ was defined empirically based on the observation of a bimodal distribution of the total distance moved in a minute.


\section*{Relative position}

\begin{align}
position &=  \frac{X - Q_{0.01}(X)}{Q_{0.99}(X - Q_{0.01}(X))}
\end{align}

Where, $Q_n$ is the quantile function.

First and last percentiles were used instead of minimum and maximum to avoid the possible effect of spurious artefactual detections beyond physical limits of the tube.
%Note that this method implies that the animals are in close proximity to each at least 1\% of observations, which



\section*{Hierarchical clustering}

\begin{align}
D(p,q) &=  \frac{\sum_{t \in T}{BD_t(p_t,q_t)}}{|T|} \\
BD_t(p_t,q_t) &= -\ln (BC(p_t,q_t))\\
BC(p_t,q_t) &= \sum_{x\in X} \sqrt{p_t(x) q_t(x)}
\end{align}


Where,
\begin{itemize}
\item $BD_t$ is the Bhattacharyya distance at a time interval $t$,
\item $T$ is the set of all tested time intervals: $T=\{[0, 0.25), [0.25,0.5), ..., [23.75, 24)\} h$,
\item $BC_t$ is the Bhattacharyya coefficient at a time interval $t$,
\item $p$ and $q$ are the observed distributions of behaviour for two different individuals, and
\item $X$ is a the set of discrete behaviours: $X = \{quiescent, micro\text{-}movement, walking\}$.
\end{itemize}

\end{document}

